\documentclass[11pt,letter]{article}
\usepackage[top=1.00in, bottom=1.0in, left=1.1in, right=1.1in]{geometry}
\renewcommand{\baselinestretch}{1.1}
\usepackage{graphicx}
\usepackage{natbib}
\usepackage{amsmath}
\usepackage{hyperref}

\def\labelitemi{--}
\parindent=0pt

\usepackage{xcolor}

\colorlet{codecolor}{black!30}
\newcommand{\codebox}[1]{%
  \colorbox{codecolor}{\ttfamily \detokenize{#1}}%
}

\begin{document}
\bibliographystyle{/Users/Lizzie/Documents/EndnoteRelated/Bibtex/styles/besjournals}
\renewcommand{\refname}{\CHead{}}

\emph{Update on working PMM model}\\

By Lizzie\\
29 June 2021\\


Let's start with the basic {\bf PMM (phylogenetic mixed model)} \\

As I understand it:
\begin{align}
y & = \alpha + \beta x + a + e\\
a & \sim normal(0, \sigma_P^2\Sigma)\\
e & \sim normal(0, \sigma_R^2I)
\end{align}
... where $\alpha$ and $\beta$, respectively, are the intercept and the slope for the co-factor $x$, $a$ is the phylogenetic random effect, and $e$ is the residual error. $\Sigma$ is a phylogenetic correlation matrix, $I$ stands for the relevant identity matrix. (The above from Chapter 11: General Quantitative Genetic Methods for Comparative Biology, by Villemereuil \& Nakagawa, though seems similar to Housworth.)\\

What we would like to do is have the phylogenetic effect structure the species-level slopes (while also allowing partial pooling on the species given uneven data, that is, for some species we might have 3 observations, each at a different $x$ value, and for other species, we have 30 observations).

We think we have this running in \verb|ubermini_2.R| (simulation code) and \verb|ubermini_2.stan|. \\

In the Stan code, $\beta$ is:\\

\verb b_force $\sim$ \verb multi_normal(rep_vector(b_z,n_sp), \\
\verb|lambda_vcv(Vphy, lam_interceptsb, sigma_interceptsb)|\\

So the \verb|b_z| represents the root trait value, and all the action happens in the second part of the function, which calls \verb|lambda_vcv|, which is defined at the top of the Stan code. This function sets up two matrices (both the size of the VCV in your data): (1) a matrix of your VCV with $\lambda$ on the off-diagonals only (\verb|local_vcv|) and (2) a matrix with $\sigma$ on the diagonal (\verb|sigma_mat|)  and then returns the product \verb|sigma_mat| * \verb|lambda_mat| * \verb|sigma_mat|. So that, when you put the  $\beta$ code above with the \verb|lambda_vcv| function you get:

\begin{equation*}
\beta \sim MVN(\mu, \Phi\Sigma)
\end{equation*}
\begin{equation*}
\Phi\Sigma = 
 \begin{bmatrix}
  \sigma^2_{\beta} &  \lambda \sigma^2_{\beta} & \lambda \sigma^2_{\beta} \\
  \lambda \sigma^2_{\beta}  & \sigma^2_{\beta} & \lambda \sigma^2_{\beta} \\
  \lambda \sigma^2_{\beta} & \lambda \sigma^2_{\beta} &   \sigma^2_{\beta}
 \end{bmatrix}
\end{equation*}

\newpage
\emph{What we tried...}\\
From May to October 2020, Lizzie worked on this model (getting help from others, with help on the matrix work especially Will Pearse). We tried a couple forms of the $MVN(\mu, \Phi\Sigma)$ (see \verb|notes/phylabtalk_oct2020| for an overview) to get to this current one. \\

It looks like it does not matter in the current test code whether or not the VCV goes in as a correlation matrix or not. This seems odd to me. \\

We have this code running on simulations with just slope (\verb|ubermini_2.R|) as well as with an intercept and up to three slopes (phylogeny on all slopes and intercept), see the simsmore folder. \\

{\bf PGLS, as I understand it} (in case you want to compare...)\\

\emph{PGLS, allowing $\lambda$ to vary} ...\\

\begin{align}
y & \sim MVN(\mu, S)\\
\mu & = \alpha +  \beta*x \\
\end{align}
where $\mu$ is a usual linear model, $y$ is reponse data (one per species), and $S$ is a covariance matrix with as many rows and columns as species. In PGLS  (allowing $\lambda$ to vary), $\Sigma$, the phylogenetic covariance matrix, is used to construct an $S$ that should look like this:
\begin{equation}
 \begin{bmatrix}
  \sigma^2 &  \lambda \sigma^2 & \lambda \sigma^2 \\
  \lambda \sigma^2  & \sigma^2 & \lambda \sigma^2 \\
  \lambda \sigma^2 & \lambda \sigma^2 &   \sigma^2
 \end{bmatrix}
\end{equation}

\end{document}
