\documentclass[12pt,a4paper]{article}

\usepackage{graphicx}
\usepackage{amsmath}
\usepackage{natbib}

\title{PMM notes}

\author{Geoffrey Legault}

\begin{document}

\section*{Methods}

\subsection*{Model}

For each of $n$ species, we assumed data were generated from the following sampling distribution:

\begin{align}
  \label{modely}
  y_j \sim \mathcal{N}(\mu_j, \sigma_e^2)
\end{align}
where
\begin{align}
  \label{modelmu}
  \mu_j = \alpha_j + \beta_{1,j} X_2 + \beta_{2,j} X_2 + \beta_{3,j} X_3
\end{align}

Predictors $X_1$, $X_2$, $X_3$ are standardized forcing, chilling, and photoperiod, and their effects on the phenology of species $j$ are determined by parameters $\beta_{1,j}$, $\beta_{2,j}$, $\beta_{3,j}$ representing phenological traits. We assumed each trait, including the specific-level intercept $\alpha_j$, evolved independently following a Brownian motion model of evolution. Thus, species traits are elements of the following normal random vectors:
\begin{align}
  \boldsymbol{\alpha} = \{\alpha_1, \ldots, \alpha_n\}^T & \text{ such that }
  \boldsymbol{\alpha} \sim \mathcal{N}(\mu_{\alpha},\boldsymbol{\Sigma_{\alpha}}) \\
  \boldsymbol{\beta_1} =  \{\beta_{1,1}, \ldots, \beta_{1,n}\}^T & \text{ such that }
  \boldsymbol{\beta_1} \sim \mathcal{N}(\mu_{\beta_1},\boldsymbol{\Sigma_{\beta_1}}) \nonumber \\
  \boldsymbol{\beta_2} =  \{\beta_{2,1}, \ldots, \beta_{2,n}\}^T & \text{ such that }
  \boldsymbol{\beta_2} \sim \mathcal{N}(\mu_{\beta_2},\boldsymbol{\Sigma_{\beta_2}}) \nonumber \\
  \boldsymbol{\beta_3} =  \{\beta_{3,1}, \ldots, \beta_{3,n}\}^T & \text{ such that }
  \boldsymbol{\beta_3} \sim \mathcal{N}(\mu_{\beta_3},\boldsymbol{\Sigma_{\beta_3}}) \nonumber
\end{align}

\noindent where the means of the multivariate normal distributions are root trait values (i.e., trait values prior to evolving across a phylogenetic tree) and $\boldsymbol{\Sigma_i}$ are $n \times n$ phylogenetic variance-covariance matrices of the form: \\

\begin{align}
  \label{phymat}
\begin{bmatrix}
  \sigma^2_i & \lambda_i \times \sigma^2_{i} \times \rho_{12} & \ldots & \lambda_i \times \sigma^2_{i} \times \rho_{1n} \\
  \lambda_i \times \sigma^2_i \times \rho_{21} & \sigma^2_i & \ldots & \lambda_i \times \sigma^2_{i} \times \rho_{2n} \\
  \vdots & \vdots & \ddots & \vdots \\
  \lambda_i \times \sigma^2_i \times \rho_{n1} & \lambda_i \times \sigma^2_i \times \rho_{n2} & \ldots & \sigma^2_i \\
\end{bmatrix}
\end{align}

where $\sigma_i^2$ is the variance of the Brownian motion model of evolution and thus represents the rate of evolution across a tree for trait $i$ (assumed to be constant along all branches). Parameter $\lambda_i$ is Pagel's $D$, which scales the covariance and therefore is a measure of the ``phylogenetic signal'' within trait $i$ (note: because estimates can be partitioned into phylogenetic and nonphylogenetic components in this model, $0 \leq \lambda \leq 1$; see Freckleton et al. 2002 Am Nat). Finally, $\rho_{xy}$ is the phylogenetic correlation between species $x$ and $y$, or the fraction of the tree shared by the two species.

The above specification is exactly equivalent to writing equation \ref{modelmu} in terms of root trait values and residuals, such that:

\begin{align}
  \mu_j = \mu_\alpha + \mu_{\beta_1} X_1 + \mu_{\beta_2} X_2 + \mu_{\beta_3} X_3 + e_{\alpha_{j}} + e_{\beta_{1,j}} + e_{\beta_{2,j}} + e_{\beta_{3,j}}
\end{align}

\noindent where the residual error terms (e.g., $e_{\alpha_{j}}$) are elements of normal random vectors from multivariate normal distributions centered on $0$ with the same phylogenetic variance-covariance matrices as in equation \ref{phymat}.

\end{document}